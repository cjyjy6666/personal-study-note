\documentclass{article}

\usepackage{amsmath}

\title{multivariable calculus}
\author{MIT18.02,fall2007}
\date{}
\begin{document}
\maketitle
\section{Dot Product}
$x+2y+3z=0$ equation of a plane\\
Let $\overrightarrow{OP}=(x,y,z)$,$\overrightarrow{A}=(1,2,3)$,then $\overrightarrow{OP}\cdot\overrightarrow{A}=0\Rightarrow P$
on a plane going through the origin point($O$).
\textbf{Remember to use vectors in multivariable calculus!}
\section{Deteminants}
\subsection{In a plane}
Area of a triangle:
\[\pm Area=\frac{1}{2}\left|\overrightarrow{A}\right|
\cdot\left|\overrightarrow{B}\right|\sin\theta
=\frac{1}{2}\left|\overrightarrow{A'}\right|\cdot\left|\overrightarrow{B}\right|\cos\theta'
=\frac{1}{2}\overrightarrow{A'}\cdot\overrightarrow{B}
=\frac{1}{2}(-a_2,a_1)\cdot(b_1,b_2)
\]
\[=\frac{1}{2}(a_1b_2-a_2b_1)
=det(A,B)
=\frac{1}{2}
\begin{vmatrix}
a_1&a_2\\
b_1&b_2
\end{vmatrix}
\]
($\overrightarrow{A'}$:$\overrightarrow{A}$ rotated by $90^\circ$ anticlockwise)
\subsection{In a space}
\subsubsection{volume of solids}
\boxed{Theorem}Geomatrically,
$det(\overrightarrow{A},\overrightarrow{B},\overrightarrow{C})
=\pm Volume\, of\, parallelepiped$

\underline{\textbf{cross product}} of 2 vectors in 3d space:\\
\boxed{Def}\[
\overrightarrow{A}\times\overrightarrow{B}
=\begin{vmatrix}
a_2&a_3\\
b_2&b_3
\end{vmatrix}\hat{i}
-\begin{vmatrix}
a_1&a_3\\
b_1&b_3
\end{vmatrix}\hat{j}
+\begin{vmatrix}
a_1&a_2\\
b_1&b_2
\end{vmatrix}\hat{k}
=\begin{vmatrix}
\hat{i} & \hat{j} & \hat{k}\\
a_1 & a_2 & a_3\\
b_1 & b_2 & b_3
\end{vmatrix}
\](3×3 only for remembering)\\
In particular,$\overrightarrow{A}\times\overrightarrow{B}
=-\overrightarrow{B}\times\overrightarrow{A}$,
and$\overrightarrow{A}\times\overrightarrow{A}=0$\\
\boxed{Theorem}
$\left|\overrightarrow{A}\times\overrightarrow{B}\right|$
is the area of the parallelogram spanned by $\overrightarrow{A}$ and $\overrightarrow{B}$.\\
$dir(\overrightarrow{A}\times\overrightarrow{B})$
is perpendicular to the plane of the parallelogram.\\
use \textbf{right-hand rule} to determine the direction of the cross product.\\
For example, $\hat{i}\times\hat{j}=\hat{k}$\\

Another loook at volumn:\\
\[Volumn=area(base)\cdot height=\left | \overrightarrow{B}\times\overrightarrow{C}  \right |\cdot(\overrightarrow{A}\cdot\vec{n})
=\left | \overrightarrow{B}\times\overrightarrow{C}  \right |\cdot(\overrightarrow{A}\cdot
\frac{\overrightarrow{B}\times\overrightarrow{C}}{\left | \overrightarrow{B}\times\overrightarrow{C}  \right |})
\]
\[=\overrightarrow{A}\cdot(\overrightarrow{B}\times\overrightarrow{C})
=det(\overrightarrow{A},\overrightarrow{B},\overrightarrow{C})\]
which makes sense with formulas.

To decide that $P$ is on the plane$P_1P_2P_3$:
\[Volumn=det(\overrightarrow{P_1P},\overrightarrow{P_1P_2},\overrightarrow{P_1P_3})=0\]
\\Other solution($\overrightarrow{N}$is a normal vector of the plane.):
\[\overrightarrow{P_1P}\perp\overrightarrow{N}\]\\
How to find a normal vector? Use the cross product of 2 vectors on the plane!\\
\[\overrightarrow{N}=\overrightarrow{P_1P_2}\times\overrightarrow{P_1P_3}\]
So the two solutions are equivalent.(triple product=det.)
\[\overrightarrow{P_1P}\cdot(\overrightarrow{P_1P_2}\times\overrightarrow{P_1P_3})
=det(\overrightarrow{P_1P},\overrightarrow{P_1P_2},\overrightarrow{P_1P_3})=0\]

\section{Matrices}
Ex.exchange of coordinate systems.\\
multiplying matrices\\
$(AB)X$\\
$AB$represents: do transformation $B$ first, then do transformation $A$.\\
Identity matrix $I$

Example:in the plane,rotation by $90^\circ$ counterclockwise:
\[use\quad R=\begin{bmatrix}
0&-1\\
1&0
\end{bmatrix}\]

Inverse matrix:for a square matrix $A$,$AA^{-1}=A^{-1}A=I$\\
So a linear system $AX=b$can be solved.\\
To find the invert matrix: adjoint matrix is introduced.
\section{Square Systems}

\end{document}