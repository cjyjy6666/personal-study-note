\documentclass{article}
\usepackage{amsmath}
\usepackage{circuitikz}
\usepackage{graphicx}
\title{Problem Set 1.1}
\author{}
\date{}

\begin{document}
\maketitle
\section{}
Under what conditions on $a,b,c,d$ is 
$\begin{bmatrix}
c\\d    
\end{bmatrix}$
a multiple m of
$\begin{bmatrix}
a\\b    
\end{bmatrix}$?
Start with the two equations$c=ma$ and $d=mb$.
By eliminating m,find one equation connecting $a,b,c,d$.
You can assume no zeroes in these numbers.
\[m=\frac{c}{a}=\frac{d}{b}
\Rightarrow ad=bc\]

\section{}
Going around a triangle from$(0,0)$to$(5,0)$to$(0,12)$to$(0,0)$,
what are those three vectors $\mathbf{u}$,$\mathbf{v}$and$\mathbf{w}$?
What is $\mathbf{u}+\mathbf{v}+\mathbf{w}$?
What are their lengths?

\[\mathbf{u}= \begin{bmatrix}
5\\0    
\end{bmatrix}\quad
\mathbf{v}= \begin{bmatrix}
-5\\12
\end{bmatrix}\quad
\mathbf{w}= \begin{bmatrix}
0\\-12
\end{bmatrix}
\]
\[\mathbf{u}+\mathbf{v}+\mathbf{w}= \begin{bmatrix}
0\\0
\end{bmatrix}
\quad
|\mathbf{u}|=5\quad
|\mathbf{v}|=13\quad
|\mathbf{w}|=12
\]
\section{}
Describe geometrically all linear combinations of the given vectors.
\\(a) $\begin{bmatrix}
    1\\2\\3
\end{bmatrix}$and$\begin{bmatrix}
    3\\6\\9
\end{bmatrix}$:a line.
(b) $\begin{bmatrix}
    1\\0\\0
\end{bmatrix}$and$\begin{bmatrix}
    0\\2\\3
\end{bmatrix}$:a plane.
\\(c) $\begin{bmatrix}
    2\\0\\0
\end{bmatrix}$and$\begin{bmatrix}
    0\\2\\2
\end{bmatrix}$and$\begin{bmatrix}
    2\\2\\3
\end{bmatrix}$:all of$R^3$.
\section{}
Draw$\mathbf{v}=\begin{bmatrix}
    4\\1
\end{bmatrix}$and $\mathbf{w}=\begin{bmatrix}
    -2\\2
\end{bmatrix}$
and $\mathbf{v}+\mathbf{w}$and $\mathbf{v}-\mathbf{w}$
in a single $xy$ plane.
\begin{figure}[!ht]
    \centering
    \resizebox{1\textwidth}{!}{%
    \begin{circuitikz}
    \tikzstyle{every node}=[font=\large]
    \draw [->, >=Stealth] (-21.25,-1) -- (-13.25,-1);
    \draw [->, >=Stealth] (-18.75,-2.25) -- (-18.75,2.75);
    \draw [->, >=Stealth] (-18.75,-1) -- (-13.75,0.25);
    \draw [->, >=Stealth] (-18.75,-1) -- (-21.25,1.5);
    \draw [->, >=Stealth] (-18.75,-1) -- (-16.25,2.75);
    \draw [->, >=Stealth] (-21.25,1.5) -- (-13.75,0.25);
    \node [font=\large] at (-12.75,-1.25) {x};
    \node [font=\large] at (-18.75,3.25) {y};
    \node [font=\large] at (-19.25,-1.25) {O};
    \node [font=\large] at (-15,-0.25) {\textbf{v}};
    \node [font=\large] at (-20.5,0.5) {\textbf{w}};
    \node [font=\large] at (-16,2.75) {\textbf{v+w}};
    \node [font=\large] at (-19.75,1.75) {\textbf{v-w}};
    \end{circuitikz}
    }%
    \end{figure}
\end{document}