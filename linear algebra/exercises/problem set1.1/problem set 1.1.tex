\documentclass{article}
\usepackage{amsmath}
\usepackage{circuitikz}
\usepackage{graphicx}
\title{Problem Set 1.1}
\author{}
\date{}

\begin{document}
\maketitle
\section{}
Under what conditions on $a,b,c,d$ is 
$\begin{bmatrix}
c\\d    
\end{bmatrix}$
a multiple m of
$\begin{bmatrix}
a\\b    
\end{bmatrix}$?
Start with the two equations$c=ma$ and $d=mb$.
By eliminating m,find one equation connecting $a,b,c,d$.
You can assume no zeroes in these numbers.
\[m=\frac{c}{a}=\frac{d}{b}
\Rightarrow ad=bc\]

\section{}
Going around a triangle from$(0,0)$to$(5,0)$to$(0,12)$to$(0,0)$,
what are those three vectors $\mathbf{u}$,$\mathbf{v}$and$\mathbf{w}$?
What is $\mathbf{u}+\mathbf{v}+\mathbf{w}$?
What are their lengths?

\[\mathbf{u}= \begin{bmatrix}
5\\0    
\end{bmatrix}\quad
\mathbf{v}= \begin{bmatrix}
-5\\12
\end{bmatrix}\quad
\mathbf{w}= \begin{bmatrix}
0\\-12
\end{bmatrix}
\]
\[\mathbf{u}+\mathbf{v}+\mathbf{w}= \begin{bmatrix}
0\\0
\end{bmatrix}
\quad
|\mathbf{u}|=5\quad
|\mathbf{v}|=13\quad
|\mathbf{w}|=12
\]
\section{}
Describe geometrically all linear combinations of the given vectors.
\\(a) $\begin{bmatrix}
    1\\2\\3
\end{bmatrix}$and$\begin{bmatrix}
    3\\6\\9
\end{bmatrix}$:a line.
(b) $\begin{bmatrix}
    1\\0\\0
\end{bmatrix}$and$\begin{bmatrix}
    0\\2\\3
\end{bmatrix}$:a plane.
\\(c) $\begin{bmatrix}
    2\\0\\0
\end{bmatrix}$and$\begin{bmatrix}
    0\\2\\2
\end{bmatrix}$and$\begin{bmatrix}
    2\\2\\3
\end{bmatrix}$:all of$R^3$.
\section{}
Draw$\mathbf{v}=\begin{bmatrix}
    4\\1
\end{bmatrix}$and $\mathbf{w}=\begin{bmatrix}
    -2\\2
\end{bmatrix}$
and $\mathbf{v}+\mathbf{w}$and $\mathbf{v}-\mathbf{w}$
in a single $xy$ plane.
\begin{figure}[h]
    \centering
    \resizebox{0.6\textwidth}{!}{%
    \begin{circuitikz}
    \tikzstyle{every node}=[font=\large]
    \draw [->, >=Stealth] (-21.25,-1) -- (-13.25,-1);
    \draw [->, >=Stealth] (-18.75,-2.25) -- (-18.75,2.75);
    \draw [->, >=Stealth] (-18.75,-1) -- (-13.75,0.25);
    \draw [->, >=Stealth] (-18.75,-1) -- (-21.25,1.5);
    \draw [->, >=Stealth] (-18.75,-1) -- (-16.25,2.75);
    \draw [->, >=Stealth] (-21.25,1.5) -- (-13.75,0.25);
    \node [font=\large] at (-12.75,-1.25) {x};
    \node [font=\large] at (-18.75,3.25) {y};
    \node [font=\large] at (-19.25,-1.25) {O};
    \node [font=\large] at (-15,-0.25) {\textbf{v}};
    \node [font=\large] at (-20.5,0.5) {\textbf{w}};
    \node [font=\large] at (-16,2.75) {\textbf{v+w}};
    \node [font=\large] at (-19.75,1.75) {\textbf{v-w}};
    \end{circuitikz}
    }%
    \end{figure}
\section{}
If $\mathbf{v}+\mathbf{w}=\begin{bmatrix}
    5\\1
\end{bmatrix}$
and $\mathbf{v}-\mathbf{w}=\begin{bmatrix}
    1\\5
\end{bmatrix}$.
Compute and draw the vectors $\mathbf{v}$and $\mathbf{w}$.
\[\mathbf{v}=\begin{bmatrix}
    3\\3
\end{bmatrix}\quad
\mathbf{w}=\begin{bmatrix}
    2\\-2
\end{bmatrix}
\]
\begin{figure}[h]
    \centering
    \resizebox{0.4\textwidth}{!}{%
    \begin{circuitikz}
    \tikzstyle{every node}=[font=\large]
    \draw [->, >=Stealth] (-21.25,-1) -- (-13.25,-1);
    \draw [->, >=Stealth] (-18.75,-2.25) -- (-18.75,2.75);
    \node [font=\large] at (-12.75,-1.25) {x};
    \node [font=\large] at (-18.75,3.25) {y};
    \node [font=\large] at (-19.25,-1.25) {O};
    \draw [short] (-18.75,-2.25) -- (-18.75,-3.5);
    \draw [->, >=Stealth] (-18.75,-1) -- (-15,2.75);
    \draw [->, >=Stealth] (-18.75,-1) -- (-16.25,-3.5);
    \node [font=\large] at (-16.25,2) {\textbf{v}};
    \node [font=\large] at (-16.75,-2.5) {\textbf{w}};
    \end{circuitikz}
    }%
    
    \label{fig:my_label}
    \end{figure}
\section{}
From$\mathbf{v}=\begin{bmatrix}
    2\\1
\end{bmatrix}$
and $\mathbf{w}=\begin{bmatrix}
    1\\2
\end{bmatrix}$,
find the components of
$3\mathbf{v}+\mathbf{w}$
and$c\mathbf{v}+d\mathbf{w}$.
\[3\mathbf{v}+\mathbf{w}=\begin{bmatrix}
    7\\5
\end{bmatrix}\quad
c\mathbf{v}+d\mathbf{w}=\begin{bmatrix}
    2c+d\\c+2d
\end{bmatrix}
\]
\section{}
$\mathbf{u}=\begin{bmatrix}
    1\\2\\3
\end{bmatrix}\quad
\mathbf{v}=\begin{bmatrix}
    -3\\1\\-2
\end{bmatrix}$
and $\mathbf{w}=\begin{bmatrix}
    2\\-3\\-1
    \end{bmatrix}$,
compute$\mathbf{u}+\mathbf{v}+\mathbf{w}$
and $2\mathbf{u}+2\mathbf{v}+\mathbf{w}$.
How do you know that $\mathbf{u},\mathbf{v},\mathbf{w}$lie in a plane?
\[\mathbf{u}+\mathbf{v}+\mathbf{w}=\begin{bmatrix}
    0\\0\\0
\end{bmatrix}\quad
2\mathbf{u}+2\mathbf{v}+\mathbf{w}=\begin{bmatrix}
    -2\\3\\1
\end{bmatrix}
\]
\[\mathbf{w}=-\mathbf{u}-\mathbf{v}
\Rightarrow rank(\mathbf{u},\mathbf{v},\mathbf{w})=2
\Rightarrow \mathbf{u},\mathbf{v},\mathbf{w} \text{lie in a plane}
\]
\section{}
Every combination of $\mathbf{v}=(1,-2,1)$
and $\mathbf{w}=(0,1,-1)$has components that add to what?
Find $c$ and $d$ so that $c\mathbf{v}+d\mathbf{w}=(3,3-6)$.
Why is $(3,3,6)$  impossible?

Combination:$a\mathbf{v}+b\mathbf{w}=(a,-2a+b,a-b)$
Their combination adds to $0$.\\
$c=3$,$d=9$
\section{}
In the $xy$ plane mark all nine of these linear combinations:\\
$c=\begin{bmatrix}
    2\\1
\end{bmatrix}$
and $d=\begin{bmatrix}
    0\\1
\end{bmatrix}$ with $c=0,1,2$ and $d=0,1,2$.
\begin{figure}[h]
    \centering
    \resizebox{0.5\textwidth}{!}{%
    \begin{circuitikz}
    \tikzstyle{every node}=[font=\small]
    \draw [->, >=Stealth] (-21.25,-1) -- (-13.25,-1);
    \draw [->, >=Stealth] (-18.75,-2.25) -- (-18.75,2.75);
    \node [font=\large] at (-12.75,-1.25) {x};
    \node [font=\large] at (-18.75,3.25) {y};
    \node [font=\large] at (-19.25,-1.25) {O};
    \draw [short] (-18.75,-2.25) -- (-18.75,-3.5);
    \draw [->, >=Stealth] (-18.75,-1) -- (-16.25,0.25);
    \draw [->, >=Stealth] (-18.75,-1) -- (-18.75,0.25);
    \node at (-18.75,-1) [circ] {};
    \draw [->, >=Stealth] (-18.75,-1) -- (-18.75,1.5);
    \draw [->, >=Stealth] (-18.75,-1) -- (-16.25,1.5);
    \draw [->, >=Stealth] (-18.75,-1) -- (-16.25,2.75);
    \draw [->, >=Stealth] (-18.75,-1) -- (-13.75,1.5);
    \draw [->, >=Stealth] (-18.75,-1) -- (-13.75,2.75);
    \draw [->, >=Stealth] (-18.75,-1) -- (-13.75,4);
    \node [font=\small] at (-18.5,-1.25) {\textit{c=0,d=0}};
    \node [font=\small] at (-19.25,-0.25) {\textit{c=0,d=1}};
    \node [font=\small] at (-19.75,1.25) {\textit{c=0,d=2}};
    \node [font=\small] at (-16.75,-0.25) {\textit{c=1,d=0}};
    \node [font=\small] at (-17,1) {\textit{c=1,d=1}};
    \node [font=\small] at (-17.25,2.5) {\textit{c=1,d=2}};
    \node [font=\small] at (-14.5,0.75) {\textit{c=2,d=0}};
    \node [font=\small] at (-13.75,2.5) {\textit{c=2,d=1}};
    \node [font=\small] at (-14,4.5) {\textit{c=2,d=2}};
    \end{circuitikz}
    }%
    \end{figure}
\section{}
How could you decide if the vectors $\mathbf{u}=(1,1,0)$
 and $\mathbf{v}=(0,1,1)$ and $\mathbf{w}=(a,b,c)$
are linearly independent or dependent?

Assume that they are linearly dependent.
Then there exist $x,y,z$ not all zero such that
$x\mathbf{u}+y\mathbf{v}+z\mathbf{w}=\mathbf{0}$.
\[x\mathbf{u}+y\mathbf{v}+z\mathbf{w}=\begin{bmatrix}
    0\\0\\0
\end{bmatrix}
\Rightarrow x(1,1,0)+y(0,1,1)+z(a,b,c)=0
\]
If $z=0$,then $x=y=0$.So$ z\neq 0$.
Let $x_1=-\frac{x}{z}$,$y_1=-\frac{y}{z}$,then
\[
\begin{cases}
a=x_1\\
b=x_1+y_1\\
c=y_1
\end{cases}
\Rightarrow b=a+c\]
\end{document}