\documentclass[10pt,a4paper]{article}
\usepackage{amsmath}
\usepackage{ctex}
\usepackage{graphicx}

\title{第八章 静电场}
\author{华中科技大学大学物理A}
\date{\today}
\begin{document}
\maketitle
\section{notes}
\subsection{电荷和库仑定律}
电量是相对论不变量

库仑定律:真空中两个点电荷相互作用力为:
\[\vec{F}=k\frac{q_1q_2}{r^2}\vec{e_r}\]
\[k=\frac{1}{4\pi\varepsilon_0}\]
注意只适用于两个静止点电荷.$\varepsilon_0$是真空介电常数.不是真空,换为$\varepsilon=\varepsilon_0\varepsilon_r$
\\库仑力通过电场传递

电场强度为:
\[\vec{E}=\frac{1}{4\pi\varepsilon}\frac{q}{r^2}\vec{e_r}\]

电偶极子.电偶极矩为$\vec{p}=q\vec{l}$,其中$\vec{l}$从负电荷指向正电荷

\textbf{电偶极子}中垂线上场强为\[E=-\frac{ql}{4\pi\varepsilon (r^2+\frac{l^2}{4})^{3/2}}\]
当$r>>l$时,
\[E=-\frac{p}{4\pi\varepsilon r^3}\]

对于\textbf{无限长的棒},距离棒$r$处的电场强度为
\[E=\frac{\lambda}{2\pi\varepsilon_0 r}\]其中$\lambda$是单位长度的电荷量

一个\textbf{无限大带电平面},面电荷密度为$\sigma$,则激发一个匀强电场:
\[E=\frac{\sigma}{2\varepsilon_0}\]

因此平行板电容器的板间电场强度为
\[E=\frac{\sigma}{\varepsilon_0}\]

一个\textbf{均匀带电圆环},半径为$R$,带电量为$Q$,则轴线上的电场强度为
\[E=\frac{xQ}{4\pi\varepsilon_0(x^2+R^2)^{3/2}}\]
若$x>>R$,则为\[E=\frac{Q}{4\pi\varepsilon_0R^2}\]

\textbf{均匀带电圆盘},电荷面密度为$\sigma$,则距圆盘$x$处的电场强度为\[E=\frac{\sigma}{2\varepsilon_0}(1-\frac{x}{\sqrt{x^2+R^2}})\]
由此也可得出无限大平面激发电场$E=\frac{\sigma}{2\varepsilon_0}$

均匀带电球面
\end{document}