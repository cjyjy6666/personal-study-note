\documentclass[10pt,a4paper]{article}
\usepackage{amsmath}
\usepackage{ctex}
\usepackage{graphicx}

\title{第七章 热力学基础}
\author{华中科技大学大学物理A}
\date{\today}
\begin{document}
\maketitle
\section{notes}
\subsection{热力学第一定律}
\[\mathrm{d} Q=\mathrm{d}E+\mathrm{d}A\]
含义分别为系统吸收的热量、系统内能的增量、系统对外做的功

在\textbf{准静态过程}中,有:
\[A=\int_{V1}^{V2} P\mathrm{d}V\]
在非静态过程中则不一定,例如气体自由膨胀时做功一定为0

热容$C$的单位为$\mathrm{J}\cdot\mathrm{K}$,定义式为
\[C=\frac{\mathrm{d}Q}{\mathrm{d}T}\]

定义摩尔热容
\[C_m=\frac{1}{\nu}\frac{\mathrm{d}Q}{\mathrm{d}T}\]
也可以定义比热容(质量热容)为单位质量气体的热容

定压热容、定容热容
\subsection{理想气体的热容}
容易推导\textbf{迈耶公式}
\[\boxed{C_{p,m}=C_{v,m}+R}\]
于是定义气体的摩尔热容比
\[\boxed{\gamma=\frac{C_{p,m}}{C_{v,m}}=1+\frac{R}{C_{v,m}}}\]
又由理想气体:
\[E_m=\frac{1}{2}(t+r+2s)RT\]
于是\[C_{v,m}=\frac{\mathrm{d}E_m}{\mathrm{d}T}=\frac{1}{2}(t+r+2s)R\]

定义物体在相变过程中吸收的热量叫做\textbf{潜热},例如熔化热和汽化热
\subsection{热力学第一定律对理想气体的应用}
理想气体所经历的\textbf{任意一个过程},内能增量均满足
\[\Delta E=\nu C_{V,m}\Delta T\]

对于\textbf{准静态}绝热过程,有绝热方程
\[pV^\gamma=C_1\]
再利用$pV=\nu RT$,也可以写成
\[TV^{\gamma-1}=C_2\]
\[p^{\gamma-1}T^{-\gamma}=C_3\]

绝热过程中功的表达式为\[A=-\Delta E=-\nu C_{V,m}\Delta T\]
对于\textbf{准静态}绝热过程,有
\[A=\frac{p_1V_1-p_2V_2}{\gamma-1}\]

可以用一个统一的公式来刻画理想气体等值过程
\[pV^n=\mathrm{const.}\]

\end{document}