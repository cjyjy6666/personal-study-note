\documentclass[10pt,a4paper]{article}
\usepackage{amsmath}
\usepackage{ctex}
\usepackage{graphicx}

\title{第七章 热力学基础}
\author{华中科技大学大学物理A}
\date{\today}
\begin{document}
\maketitle
\section{notes}
\subsection{热力学第一定律}
\[\mathrm{d} Q=\mathrm{d}E+\mathrm{d}A\]
含义分别为系统吸收的热量、系统内能的增量、系统对外做的功

在\textbf{准静态过程}中,有:
\[A=\int_{V1}^{V2} P\mathrm{d}V\]
在非静态过程中则不一定,例如气体自由膨胀时做功一定为0

热容$C$的单位为$\mathrm{J}\cdot\mathrm{K}$,定义式为
\[C=\frac{\mathrm{d}Q}{\mathrm{d}T}\]

定义摩尔热容
\[C_m=\frac{1}{\nu}\frac{\mathrm{d}Q}{\mathrm{d}T}\]
也可以定义比热容(质量热容)为单位质量气体的热容

定压热容、定容热容
\subsection{理想气体的热容}
容易推导\textbf{迈耶公式}
\[\boxed{C_{p,m}=C_{v,m}+R}\]
于是定义气体的摩尔热容比
\[\boxed{\gamma=\frac{C_{p,m}}{C_{v,m}}=1+\frac{R}{C_{v,m}}}\]
又由理想气体:
\[E_m=\frac{1}{2}(t+r+2s)RT\]
于是\[C_{v,m}=\frac{\mathrm{d}E_m}{\mathrm{d}T}=\frac{1}{2}(t+r+2s)R\]

定义物体在相变过程中吸收的热量叫做\textbf{潜热},例如熔化热和汽化热
\subsection{热力学第一定律对理想气体的应用}
理想气体所经历的\textbf{任意一个过程},内能增量均满足
\[\Delta E=\nu C_{V,m}\Delta T\]

对于\textbf{准静态}绝热过程,有绝热方程
\[pV^\gamma=C_1\]
再利用$pV=\nu RT$,也可以写成
\[TV^{\gamma-1}=C_2\]
\[p^{\gamma-1}T^{-\gamma}=C_3\]

绝热过程中功的表达式为\[A=-\Delta E=-\nu C_{V,m}\Delta T\]
对于\textbf{准静态}绝热过程,有
\[A=\frac{p_1V_1-p_2V_2}{\gamma-1}\]

可以用一个统一的公式来刻画理想气体等值过程
\[pV^n=\mathrm{const.}\]
上式中$n=0$对应等压过程,$n\rightarrow\infty$对应等容过程,$n=\gamma$为绝热过程\\
若$0<n<\gamma$则为\textbf{多方过程},常数$n$称为多方指数
\subsection{循环过程、卡诺循环}
工质

热机效率(循环效率):
\[\eta=\frac{A}{Q_1}=\frac{Q_1-Q_2}{Q_1}=1-\frac{Q_2}{Q_1}\]
其中$Q_1$为从高温热源吸热,$Q_2$为向低温热源放热

\textbf{卡诺循环}:两个等温过程和两个绝热过程,循环效率为:
\[\boxed{\eta=1-\frac{T_2}{T_1}}\]
卡诺循环是高温热源($T_1$)和低温热源($T_2$)的之间效率最高的循环

同样还有制冷机。定义制冷系数为:
\[w=\frac{Q_2}{A}=\frac{Q_2}{Q_1-Q_2}\]
其中$Q_2$为从低温热源吸热,$Q_1$为向高温热源放热

卡诺制冷机的制冷系数:
\[w=\frac{T_2}{T_1-T_2}\]
\subsection{热力学第二定律}
可逆过程要求系统周围一切也各自恢复原状

\textbf{无摩擦的准静态过程是可逆过程}

不可逆过程:系统恢复不了原态;恢复原态却引起了外界的变化

热力学第二定律

开尔文表述:不可能从单一热源吸收热量并将其完全转化为功(第二类永动机)

克劳修斯表述:不可能使热量自发地从低温物体传递到高温物体

卡诺定理:可逆卡诺热机的效率最高

可逆循环中有$\oint\frac{\mathrm{d}Q}{T}=0$
\subsection{熵Entropy}
克劳修斯熵公式(热温比)
\[\mathrm{d}S=\frac{\mathrm{d}Q}{T}\]
\textbf{注意仅对可逆过程成立}。对于不可逆过程,可以构造连接始末状态的可逆过程

熵增原理:
\[\text{可逆过程}\mathrm{d}S=\mathrm{d}Q/T\]
\[\text{不可逆过程}\mathrm{d}S>\mathrm{d}Q/T\]
于是在\textbf{孤立(或绝热)系统}中$\Delta S\geq 0$,其中$=$对应可逆过程,这是热力学第二定律的数学表达

温熵图($T-S$图)
\[\mathrm{d}Q=T\mathrm{d}S\]
于是温熵图曲线下面积为吸收的热量,闭合曲线内的面积为做功

热力学概率$\Omega$为宏观态对应的微观态的数目,是宏观态的函数

玻尔兹曼熵
\[S=k\mathrm{ln}\Omega\]
对应系统的非平衡态也有熵,意义更为普遍

因平衡态的热力学概率最大,因此克劳修斯熵是玻尔兹曼熵的最大值

熵具有可加性
\end{document}