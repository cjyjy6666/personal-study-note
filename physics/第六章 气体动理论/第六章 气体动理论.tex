\documentclass[10pt,a4paper]{article}
\usepackage{amsmath}
\usepackage{ctex}
\usepackage{graphicx}

\title{第六章 气体动理论}
\author{华中科技大学大学物理A}
\date{\today}
\begin{document}
\maketitle
\section{notes}
孤立、封闭、开放系统
\subsection{理想气体状态方程与微观模型}
理想气体物态方程:
\[pV=\nu RT\]
其中$R=8.13J/mol\cdot K$

令$k=\frac{R}{N_A}=1.38\times10^{-23}J/K$为玻尔兹曼常数,于是方程可以写成$pV=NkT$或:
\[\boxed{p=nkT}\]
其中$n=\frac{N}{V}$为分子密度

\textit{混合气体的物态方程}

道尔顿分压定律:压强为分压强之和\\
写成物态方程:$n$为各组之和,$M$用平均摩尔质量

\textit{理想气体压强}
\[P=\frac{2}{3}n\overline{\varepsilon_{kt}}\]
其中分子平动动能平均值
\[\overline{\varepsilon_{kt}}=\frac{1}{2}m_f\bar{v^2}\]
于是结合状态方程有:
\[\overline{\varepsilon_t}=\frac{3}{2}kT\]
由$\frac{1}{2}m_f\bar{v^2}=\frac{3}{2}kT$得方均根(RMS)速率
\[\sqrt{\overline{v^2}}=\sqrt{\frac{3kT}{m_f}}=\sqrt{\frac{3RT}{M}}\]
\subsection{能均分定理与理想气体内能}
气体分子自由度$i$

\begin{itemize}
\item 单原子分子:$t_{\text{平动}}=3$,$i=3$
\item 刚性双原子分子:$t_{\text{平动}}=3$,$r_{\text{转动}}=2$,$i=5$
\item 非刚性双原子分子:$t_{\text{平动}}=3$,$r_{\text{转动}}=3$,$s_{\text{振动}}=1$,$i=6$
\item 刚性多原子分子:$t_{\text{平动}}=3$,$r_{\text{转动}}=3$,$i=6$
\item 非刚性多原子分子:$t_{\text{平动}}=3$,$r_{\text{转动}}=3$,$s_{\text{振动}}=3n-6$,$i=3n$
\end{itemize} 
\textbf{能均分定理}:每个自由度都具有$\frac{1}{2}kT$的平均动能\\
于是具有$i$个自由度的分子总平均动能为$\overline{\varepsilon_k}=\frac{i}{2}kT$

另外每个分子还有势能$\varepsilon_p$
\[\overline{\epsilon}=\overline{\epsilon_k}+\overline{\epsilon_p}=\frac{i}{2}kT+\frac{s}{2}kT\]

内能是温度的单值函数!
\subsection{实际气体物态方程}
\textbf{Van Der Waals equation}

$1mol$气体:
\[(p+\frac{a}{V_m^2})(V_m-b)=RT\]
一般的:
\[(p+\frac{m^2}{M^2}\frac{a}{V^2})(V-\frac{m}{M}b)=\frac{m}{M}RT\]
\subsection{麦克斯韦速率分布}
速率分布函数$f(v)$有
\[\boxed{f(v)dv=\frac{dN}{N}}\]
\[f(v)=4\pi\left(\frac{m_f}{2\pi kT}\right)^{\frac{3}{2}}e^{-\frac{m_fv^2}{2kT}}v^2\]

速率分布函数的归一化条件
\[\int_0^{+\infty}f(v)dv=1\]
\textbf{分子的三个特征速率}

\textit{平均速率}(所有分子速率算数平均值)
\[\bar{v}=\sqrt{\frac{8kT}{\pi m_f}}=\sqrt{\frac{8RT}{\pi M}}\]

\textit{方均根速率}(RMS vetolocity)
\[v_{rms}=\sqrt{\frac{3kT}{m_f}}=\sqrt{\frac{3RT}{M}}\]

\textit{最概然速率}(when $f(v)$ reaches max)
\[v_{p}=\sqrt{\frac{2kT}{m_f}}=\sqrt{\frac{2RT}{M}}\]

relationship:
\[v_p<\bar{v}<v_{rms}\]
\subsection{Maxwell\&Boltzmann distribution functions}
麦克斯韦速度分布律给出速度区间内分子数占总分子数的比例
\[\frac{dN_v}{N}=\left(\frac{m_f}{2\pi kT}\right)^{\frac{3}{2}e^{-m_f(v_x^2+v_y^2+v_z^2)/(2kT)}} \]
麦克斯韦速度分布函数
\[f(\vec{v})=\left(\frac{m_f}{2\pi kT}\right)^{\frac{3}{2}e^{-m_f(v_x^2+v_y^2+v_z^2)/(2kT)}}\]
Boltzmann分布:在力场中,$E_p$为势能,则分子数密度有
\[\boxed{n=n_0e^{-\frac{E_p}{kT}}}\]
\subsection{分子的平均碰撞次数与平均自由程}
def平均自由程$\bar{\lambda}$:

气体分子是不断变化的

def平均碰撞频率$\bar{Z}$:一个分子单位时间里收到的平均碰撞次数

\[\bar{\lambda}=\frac{kT}{\sqrt{2}\pi d^2P}\]
\[\bar{Z}=\sqrt{2}n\pi d^2\bar{v}\]
\subsection{输运现象}
输运过程:不受外界干扰时系统总要从非平衡态自发地向平衡态过渡

\textbf{内摩擦(粘滞现象)}
动量的输运

产生原因:系统内各部分流速不同
\[df=-\eta\frac{du}{dy}dS\]
内摩擦系数$\eta$单位为Pa$\cdot$s,表达式为
\[\eta=\frac{1}{3}nm_f\bar{v}\bar{\lambda}\]

\textbf{热传导}

能量的输运

系统内各部分温度不同
\[dQ=-\kappa \frac{dT}{dy}dsdt\]
这是$dt$时间内通过面积$ds$沿着$y$轴方向传递的热量

\textbf{气体的扩散}

质量的输运

系统内各部分密度不同
\[dm=-D\frac{d\rho}{dy}dsdt\]
为$dt$时间内通过面积$ds$从密度较大的一侧向密度较小的一侧扩散的气体质量
\end{document}