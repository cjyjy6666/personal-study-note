\documentclass{article}

\usepackage{amsmath}
\usepackage{ctex}

\title{第一章 质点运动学}
\author{华中科技大学大学物理A}
\date{2025.2.21}
\begin{document}
\maketitle
\noindent 参考系:参照物+坐标系+时钟\\
可作质点的条件:1.不变形、不转动 2.本身线度$<<$活动范围\\
否则:微元法 1.质点系 2.连续体切为质量元


\noindent 位矢$\vec{r}=x\vec{i}+y\vec{j}+z\vec{k}$\\
运动方程$\vec{r}(t)$,消去$t$得轨迹方程\\
位移大小记为$\lvert \Delta\vec{r} \rvert$,而\underline{$\Delta r$表示的是位矢长度的增量}\\
\[\Delta s\geq \lvert \Delta\vec{r} \rvert \geq \Delta r
\footnote{做不回头的一维直线运动时取等}\]\\
且有当$\Delta t\to0$时,$ds=\lvert d\vec{r} \rvert$\\
自然坐标系:$\hat{\mathbf{e}_\tau}\,\hat{\mathbf{e}_n}$
\[\vec{v}=v\hat{\mathbf{e}_\tau}\quad
\vec{a}=\frac{d}{dt}(v\hat{\mathbf{e}_\tau})
=\frac{dv}{dt}\hat{\mathbf{e}_\tau}
+v\frac{d\hat{\mathbf{e_\tau}}}{dt}\]
\[\vec{a_\tau}=\frac{dv}{dt}\hat{\mathbf{e}_\tau}\quad
\quad\vec{a_n}=\frac{v^2}{\rho}\hat{\mathbf{e}_n}\]
\\例:$\vec{r}=t^2\vec{i}+(2t-1)\vec{j}$
求$\lvert \vec{a_n} \rvert$.
\[\vec{v}=\frac{d\vec{r}}{dt}\quad
\vec{a}=\frac{d\vec{v}}{dt}\quad\lvert \vec{a} \rvert=\sqrt{\lvert \vec{a} \rvert^2-\lvert \vec{a_\tau} \rvert^2}
\]
\begin{center}
不一定直接求曲率半径$\rho$
\end{center}
\newpage
\noindent 求$\vec{a}(\vec{v})\quad\vec{a}(\vec{r})$:分离变量,积分\\
例:已知$a=kx$,求$v(x)$
\[
kx=a=\frac{dv}{dt}=\frac{dv}{dx}\frac{dx}{dt}=\frac{dv}{dx}v
\]
\[\Rightarrow \int kx\,dx=\int v\,dv
\Rightarrow v=\sqrt{kx^2+C}
\]
相对运动\footnote{隐含前提条件:绝对时空观}\\
\[\vec{v_{\text{绝对}}}=\vec{v_{\text{相对}}}+\vec{v_{\text{牵连}}}\]
\[\vec{a_{\text{绝对}}}=\vec{a_{\text{相对}}}+\vec{a_{\text{牵连}}}\]
\end{document}

