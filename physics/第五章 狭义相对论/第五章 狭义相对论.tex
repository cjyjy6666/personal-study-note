\documentclass[10pt,a4paper]{article}
\usepackage{amsmath}
\usepackage{ctex}
\usepackage{graphicx}

\title{第五章 狭义相对论}
\author{华中科技大学大学物理A}
\date{\today}
\begin{document}
\maketitle
\section{notes}
\subsection{狭义相对论时空观}
同时性的相对性:沿两惯性系相对运动方向发生的两个事件,在其中一个惯性系中表现同时,在另一惯性系中观察总是在前一惯性系运动的\textbf{后方那一事件先发生}

时间膨胀(\textbf{钟慢效应})
\[\Delta t=\frac{\Delta t'}{\sqrt{1-(\frac{v}{c})^2}}\]
其中原时最短.def\textbf{原时}(本征时间):某一参考系同一地点先后发生的两个事件之间的时间间隔

长度收缩(\textbf{尺缩效应})
\[L=L_0\sqrt{1-(\frac{v}{c})^2}\]
def原长:物体相对参考系静止时测得的长度\,原长最长!
\subsection{洛伦兹变换}
引入$\beta=\frac{v}{c}\quad\gamma=\frac{1}{\sqrt{1-\beta^2}}$,
则有洛伦兹变换的正变换以及逆变换:
\[
\begin{cases}
x'=\gamma(x-vt)\\
y'=y\\
z'=z\\
t'=\gamma(t-\frac{v}{c^2}x)
\end{cases}
\quad
\begin{cases}
x=\gamma(x'+vt')\\
y=y'\\
z=z'\\
t=\gamma(t'+\frac{v}{c^2}z')
\end{cases}
\]
由此也可推出洛伦兹速度变换.
\subsection{狭义相对论动力学}
相对论质量:
\[m=\gamma m_0=\frac{m_0}{\sqrt{1-(\frac{v}{c})^2}}\]
于是动量:
\[p=\gamma m_0v\]
相对论能量
\[E=mc^2\]
相对论动能:
\[E_k=mc^2-m_0c^2=(\gamma-1)m_0c^2\]
即
\[\text{动能}=\text{总能量}-\text{静能}\]
相对论动量-能量关系(关系为直角三角形三边长关系):
\[E^2=(pc)^2+(m_0c^2)^2\]
\end{document}