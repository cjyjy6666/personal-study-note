\documentclass[10pt,a4paper]{article}
\usepackage{amsmath}
\usepackage{ctex}
\usepackage{graphicx}

\title{第三章 刚体}
\author{华中科技大学大学物理A}
\date{\today}
\begin{document}
\maketitle
\section{notes}
\subsection{刚体的平动和转动}
def刚体:物体上各点的相对位置不变,受力时大小和形状都保持不变.

def刚体平动:任两点连线运动时平行,各质点运动状态完全相同.
\\于是用质心来代表整体运动.质心可能既无质量、又未受力.
质心计算:
\[
\vec{r_c}=\frac{\sum m_i\vec{r_i}}{\sum m_i}
=\frac{\int\vec{r}dm}{M}
\]
可在分量上计算质心坐标值.刚体密度均匀、形状对称,则质心即为几何对称中心.
注意质心$\neq$重心,重心只有在重力场中才存在.
\\质心速度$\vec{v_c}=\frac{\sum m_i\vec{v_i}}{M}$
,质心加速度$\vec{a_c}=\frac{\sum m_i\vec{a_i}}{M}$
,质心运动定理:$\vec{F_\text{合外}}=M\vec{a_c}$

{\fontsize{8pt}{8pt}\selectfont
例:求半径为$R$的半圆形均匀铁丝质心.
\[\lambda=\frac{m}{\pi R}\quad dm=\lambda d\theta R=\frac{m}{\pi}d\theta
\]
\[x_c=\frac{\int xdm}{m}=\frac{\int_0^\pi R\cos\theta md\theta}{\pi m}=0
\quad y_c=\frac{\int ydm}{m}=\frac{\int_0^\pi R\sin\theta md\theta}{\pi m}=\frac{2}{\pi}R\]

故质心$(0,\frac{2}{\pi}R)$
}

def定轴转动:转轴位置、方向固定不变.\\
角位置$\vec{\theta}$,角速度$\vec{\omega}$,角加速度$\vec{\alpha}$.\\
角量与线量:
\[\vec{v}=\vec{\omega}\times\vec{r}\]
\[\boxed{\vec{a_{\tau}}=\vec{\alpha}\times\vec{r}}\quad\vec{a_n}=\vec{\omega}\times\vec{v}\]
\[\vec{a}=\vec{\alpha}\times\vec{r}+\vec{\omega}\times\vec{v}\]
\subsection{刚体定轴转动定律}
$\vec{F}=\vec{F_{//}}+\vec{F_\perp}$,定轴转动时只需考虑$\vec{F_{\perp}}$(垂直于转轴的分力)的力矩$M_z$
\\力$F$相对于转轴($z$)的力矩:
\[\vec{M_z}=\vec{r}\times\vec{F_\perp}\quad M_z=F_\perp r\sin\theta\]
其中$r\sin\theta$即为力臂.

{
\fontsize{8pt}{8pt}\selectfont
刚体上质元的角动量$\vec{L_i}=\vec{R_i}\times m_i\vec{v_i}=m_i(z_i\vec{k}-r_i\vec{n}\times r_i\omega\vec{\tau_i})
=m_i\omega r_i(z_i\vec{k}\times \vec{\tau_i}+r_i\vec{k})$,则
$M_z=\vec{M}\cdot\vec{k}=\frac{d\sum \vec{L_i}}{dt}\cdot\vec{k}=\frac{d(\sum m_ir_i^2\omega)}{dt}=\alpha\sum m_ir_i^2$
}

let:$J=\sum m_ir_i^2$(转动惯量),则
\[\boxed{\vec{M}=J\vec{\alpha}}\]这是\textbf{刚体定轴转动定律}.角动量的$z$分量为刚体对$z$轴的角动量,$\boxed{L=J\omega}$
\\$J$反应刚体的转动惯性.总质量一定,质量分布离轴越远,$J$越大.
\subsection{转动惯量的计算}
\[\boxed{J=\sum m_ir_i^2=\int r^2dm}\]单位$kg\cdot m^2$.
实际计算时考虑$dm=\lambda dl$,$dm=\sigma dS$,$dm=\rho dV$.

\textit{常见刚体的转动惯量}

均匀圆环或圆筒
\[J_c=mR^2\]

均匀圆盘或圆柱绕盘心转动
\[J_c=\frac{1}{2}mR^2\]

均匀圆盘或圆柱绕一端转动
\[J_c=\frac{1}{2}mR^2+mR^2\]

均匀杆绕一端转动
\[J_c=\frac{1}{3}mL^2\]

均匀杆绕中点转动
\[J_c=\frac{1}{12}mL^2\]

薄球壳
\[J_c=\frac{2}{3}mR^2\]

球体
\[J_c=\frac{2}{5}mR^2\]

圆筒(厚度不忽略)
\[J_c=\frac{m}{2}(R_1^2+R_2^2)\]

\textbf{平行轴定理}
\[\boxed{J=J_c+md^2}\]

如何选取质量元?看离转轴的距离!
\end{document}