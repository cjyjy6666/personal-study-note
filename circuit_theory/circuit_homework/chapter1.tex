\documentclass{article}
\usepackage{amsmath}
\usepackage{ctex}
\usepackage{graphicx}
\usepackage{circuitikz}

\title{电路第一章作业}
\author{华中科技大学电路理论(五)}
\date{}
\begin{document}
\maketitle
\noindent\boxed{1-1}
说明图a,b中:(1)$u$,$i$的参考方向是否关联?(2)$ui$乘积表示什么功率?
(3)如果在图(a)中$u>0$,$i<0$,图(b)中$u>0$,$i>0$,
元件实际发出还是吸收功率?
\begin{figure}[!ht]
    \centering
    \resizebox{0.4\textwidth}{!}{%
    \begin{circuitikz}
    \tikzstyle{every node}=[font=\small]
    \draw (1.5,6.5) to[short, -o] (0.75,6.5) ;
    \draw  (1.5,6.75) rectangle (2,6.25);
    \draw (2,6.5) to[short, -o] (2.75,6.5) ;
    \draw [->, >=Stealth] (1,6.5) -- (1.25,6.5);
    \node [font=\small] at (1,6.75) {i};
    \node [font=\small] at (1.75,6) {u};
    \node [font=\small] at (0.75,6) {+};
    \node [font=\small] at (2.75,6) {-};
    \node [font=\small] at (1.75,7) {元件};
    \draw (4,6.5) to[short, -o] (3.25,6.5) ;
    \draw  (4,6.75) rectangle (4.5,6.25);
    \draw (4.5,6.5) to[short, -o] (5.25,6.5) ;
    \node [font=\small] at (3.25,6) {+};
    \node [font=\small] at (5.25,6) {-};
    \node [font=\small] at (3.5,6.75) {i};
    \node [font=\small] at (4.25,6) {u};
    \node [font=\small] at (4.25,7) {元件};
    \node [font=\small] at (1.75,5.75) {(a)};
    \draw [->, >=Stealth] (3.75,6.5) -- (3.5,6.5);
    \node [font=\small] at (4.25,5.75) {(b)};
    \end{circuitikz}
    }%
    \end{figure}

\noindent 解:(a)关联;$ui$表示吸收功率;元件实际发出功率;\\
(b)非关联;$ui$表示发出功率;元件实际发出功率。

\noindent\boxed{1-3}
求解电路以后,校验所得结果的方法之一是核对电路中所有元件的功率平衡,
即一部分元件发出的总功率应等于其它元件吸收的总功率。
试校核图中电路所得解答是否正确。
\begin{figure}[!ht]
    \centering
    \resizebox{0.4\textwidth}{!}{%
    \begin{circuitikz}
    \tikzstyle{every node}=[font=\small]
    \draw (0.75,6.25) to[short] (0.75,7);
    \draw (0.75,7) to[short] (1.5,7);
    \draw (1.5,7) to[short] (1.5,6.25);
    \draw  (0.5,6.25) rectangle (1,5.75);
    \draw  (1.25,6.25) rectangle (1.75,5.75);
    \draw [short] (0.75,5.75) -- (0.75,5.25);
    \draw [short] (1.5,5.75) -- (1.5,5.25);
    \draw [short] (0.75,5.25) -- (2.25,5.25);
    \draw [short] (2.25,5.25) -- (2.25,5.75);
    \draw [short] (2.25,6.25) -- (2.25,7);
    \draw [short] (1.5,7) -- (2.75,7);
    \draw  (2.75,7.25) rectangle (3.25,6.75);
    \draw [short] (3.25,7) -- (3.75,7);
    \draw [short] (3.75,7) -- (3.75,6.25);
    \draw [short] (2.25,5.25) -- (3.75,5.25);
    \draw [short] (3.75,5.25) -- (3.75,5.75);
    \draw  (2,6.25) rectangle (2.5,5.75);
    \draw  (3.5,6.25) rectangle (4,5.75);
    \node at (1.5,7) [circ] {};
    \node at (2.25,7) [circ] {};
    \node at (2.25,5.25) [circ] {};
    \node [font=\small] at (0.5,6.5) {+};
    \node [font=\footnotesize] at (0.25,6) {60};
    \node [font=\footnotesize] at (0.25,5.75) {V};
    \node [font=\small] at (0.5,5.5) {-};
    \node at (1.5,5.25) [circ] {};
    \draw [->, >=Stealth] (0.75,6.5) -- (0.75,6.75);
    \draw [->, >=Stealth] (1.5,6.75) -- (1.5,6.5);
    \draw [->, >=Stealth] (2.25,6.75) -- (2.25,6.5);
    \draw [->, >=Stealth] (3.5,7) -- (3.75,7);
    \node [font=\normalsize] at (0.75,6) {A};
    \node [font=\normalsize] at (1.5,6) {B};
    \node [font=\normalsize] at (2.25,6) {C};
    \node [font=\normalsize] at (3,7) {D};
    \node [font=\normalsize] at (3.75,6) {E};
    \node [font=\footnotesize] at (0.5,6.75) {5A};
    \node [font=\footnotesize] at (1.25,6.75) {1A};
    \node [font=\footnotesize] at (2,6.75) {2A};
    \node [font=\footnotesize] at (3.75,7.25) {2A};
    \node [font=\footnotesize] at (1.25,6.5) {+};
    \node [font=\small] at (1.25,5.5) {-};
    \node [font=\small] at (1.25,6) {60};
    \node [font=\small] at (1.25,5.75) {V};
    \node [font=\small] at (2.5,6.5) {+};
    \node [font=\small] at (2.5,5.5) {-};
    \node [font=\small] at (2.75,6) {60V};
    \node [font=\small] at (3,7.5) {+ 40V -};
    \node [font=\small] at (3.5,6.5) {+};
    \node [font=\small] at (3.5,5.5) {-};
    \node [font=\small] at (4.25,6) {20V};
    \end{circuitikz}
    }%
    \end{figure}

\noindent 解:$P_\text{发出}=300W\quad 
P_\text{吸收}60W+120W+80W=300W$\\
$P_\text{发出}=P_\text{吸收}$,满足功率平衡。

\noindent\boxed{1-4}
在指定的电压$u$和电流$i$的参考方向下,写出图中各元件的$u$和$i$
的约束方程(即$VCR$)。
\begin{figure}[!ht]
    \centering
    \resizebox{0.6\textwidth}{!}{%
    \begin{circuitikz}
    \tikzstyle{every node}=[font=\small]
    \draw (0,7.25) to[short, -o] (0,7.25) ;
    \draw (0.25,7.25) to[short, -o] (-0.25,7.25) ;
    \draw (0.25,7.25) to[european resistor] (2.25,7.25);
    \draw (2.25,7.25) to[short, -o] (2.75,7.25) ;
    \draw (4,7.25) to[short, -o] (3.25,7.25) ;
    \draw (4,7.25) to[european resistor] (6,7.25);
    \draw (6,7.25) to[short, -o] (6.5,7.25) ;
    \draw (7.75,7.25) to[short, -o] (7,7.25) ;
    \draw (9.75,7.25) to[short, -o] (10.25,7.25) ;
    \draw (0.25,6.5) to[short, -o] (-0.25,6.5) ;
    \draw (2.25,6.5) to[short, -o] (2.75,6.5) ;
    \draw (4,6.5) to[short, -o] (3.25,6.5) ;
    \draw (6,6.5) to[short, -o] (6.5,6.5) ;
    \draw (7.75,6.5) to[short, -o] (7,6.5) ;
    \draw (9.75,6.5) to[short, -o] (10.25,6.5) ;
    \draw [->, >=Stealth] (0,7.25) -- (0.25,7.25);
    \draw [->, >=Stealth] (0,6.5) -- (0.25,6.5);
    \node [font=\small] at (1.25,7.75) {10k$\Omega$};
    \node [font=\footnotesize] at (1.25,7) {u};
    \node [font=\footnotesize] at (0.25,7.5) {i};
    \node [font=\footnotesize] at (-0.25,7.5) {+};
    \node [font=\footnotesize] at (2.75,7.5) {-};
    \node [font=\small] at (2.5,7.75) {(a)};
    \node [font=\footnotesize] at (0.25,6.75) {i};
    \node [font=\footnotesize] at (-0.25,6.25) {+};
    \node [font=\footnotesize] at (2.75,6.25) {-};
    \draw [short] (0.25,6.5) -- (2.25,6.5);
    \draw  (1.25,6.5) circle (0.25cm);
    \node [font=\footnotesize] at (1.25,6.75) {u};
    \node [font=\small] at (1.25,6) {5V};
    \node [font=\small] at (0.75,6) {-};
    \node [font=\small] at (1.75,6) {+};
    \node [font=\small] at (2.5,6.75) {(d)};
    \node [font=\small] at (6.5,7.75) {(b)};
    \node [font=\small] at (6.5,7.5) {-};
    \node [font=\small] at (3.25,7.5) {+};
    \node [font=\small] at (3.25,6.25) {+};
    \node [font=\small] at (6.5,6.25) {-};
    \draw [short] (4,6.5) -- (4.5,6.5);
    \draw [short] (6,6.5) -- (5.5,6.5);
    \draw  (5,6.5) circle (0.25cm);
    \draw [short] (5.5,6.5) -- (5.25,6.5);
    \draw [short] (4.5,6.5) -- (4.75,6.5);
    \draw [short] (5,6.75) -- (5,6.25);
    \node [font=\small] at (5,6) {u};
    \node [font=\small] at (5.75,6) {10mA};
    \node [font=\small] at (5,7.75) {10$\Omega$};
    \draw [->, >=Stealth] (3.5,7.25) -- (4,7.25);
    \node [font=\small] at (4,7.5) {i};
    \node [font=\small] at (5.5,7) {u};
    \node [font=\small] at (4,6.75) {i};
    \node [font=\small] at (6.5,6.75) {(e)};
    \draw [->, >=Stealth] (3.5,6.5) -- (4,6.5);
    \draw [short] (7.75,7.25) -- (8,7.25);
    \draw [short] (9.75,7.25) -- (9.5,7.25);
    \draw [short] (8,7.25) -- (8.25,7.25);
    \draw [short] (9.5,7.25) -- (9.25,7.25);
    \draw [short] (7.75,6.5) -- (8.25,6.5);
    \draw [short] (9.75,6.5) -- (9.25,6.5);
    \draw  (8.75,7.25) circle (0.25cm);
    \draw  (8.75,6.5) circle (0.25cm);
    \draw [short] (8.25,7.25) -- (8.5,7.25);
    \draw [short] (8.25,6.5) -- (8.5,6.5);
    \draw [short] (9,6.5) -- (9.25,6.5);
    \draw [short] (9.25,7.25) -- (9,7.25);
    \draw [short] (9,7.25) -- (8.5,7.25);
    \draw [short] (8.75,6.75) -- (8.75,6.25);
    \node [font=\small] at (7,7.5) {+};
    \node [font=\small] at (10.25,7.5) {-};
    \node [font=\small] at (10.25,6.25) {+};
    \node [font=\small] at (7,6.25) {-};
    \node [font=\small] at (7.75,7.5) {i};
    \node [font=\small] at (7.75,6.25) {i};
    \node [font=\small] at (8.75,7.75) {u};
    \node [font=\small] at (8.75,6) {u};
    \node [font=\small] at (10.25,7.75) {(c)};
    \node [font=\small] at (10.25,6.75) {(f)};
    \node [font=\small] at (9.25,7.5) {10V};
    \node [font=\small] at (9.5,6.25) {10mA};
    \draw [->, >=Stealth] (7.5,7.25) -- (7.75,7.25);
    \draw [->, >=Stealth] (8,6.5) -- (7.5,6.5);
    \draw [->, >=Stealth] (9.25,6.5) -- (9.5,6.5);
    \end{circuitikz}
    }%
    \end{figure}

\noindent 解:(a)$u=iR=1\times10^4i$\quad 
(b)$u=iR=10i$\quad
(c)$u=10V$,$i$不定\\
(d)$u=-5V$,$i$不定\quad
(e)$i=10mA$,$u$不定\quad
(f)$i=-10mA$,$u$不定

\noindent\boxed{1-5}
试求图中各电路中电压源、电流源及电阻的功率(须说明是吸收还是发出)。
\begin{figure}[!ht]
    \centering
    \resizebox{0.8\textwidth}{!}{%
    \begin{circuitikz}
    \tikzstyle{every node}=[font=\small]
    \draw [short] (0.25,7.25) -- (0.25,6.5);
    \draw  (0.25,6.25) circle (0.25cm);
    \draw [short] (0.25,6) -- (0.25,5.25);
    \draw [short] (0.25,7.25) -- (0.75,7.25);
    \draw (0.75,7.25) to[european resistor] (2.75,7.25);
    \draw (2.75,7.25) to[short] (3.25,7.25);
    \draw (3.25,7.25) to[short] (3.25,6.5);
    \draw (0.25,5.25) to[short] (3.25,5.25);
    \draw (3.25,5.25) to[short] (3.25,6);
    \draw  (3.25,6.25) circle (0.25cm);
    \draw (0,6.25) to[short] (0.5,6.25);
    \draw (3.25,6.5) to[short] (3.25,6);
    \node [font=\small] at (1.75,6.75) {5$\Omega$};
    \node [font=\small] at (0.75,5.75) {2A};
    \node [font=\small] at (2.75,6.25) {15V};
    \node [font=\small] at (3,6.75) {+};
    \node [font=\small] at (3,5.75) {-};
    \node [font=\small] at (1.75,4.75) {(a)};
    \draw (4.25,7.25) to[short] (4.25,6.5);
    \draw (4.25,5.25) to[short] (4.25,6);
    \draw (4.25,7.25) to[short] (5.75,7.25);
    \draw (5.75,7.25) to[european resistor] (5.75,5.25);
    \draw (4.25,5.25) to[short] (7.25,5.25);
    \draw (7.25,5.25) to[short] (7.25,6);
    \draw (5.75,7.25) to[short] (7.25,7.25);
    \draw (7.25,7.25) to[short] (7.25,6.75);
    \draw  (4.25,6.25) circle (0.25cm);
    \draw  (7.25,6.25) circle (0.25cm);
    \draw (7.25,6.5) to[short] (7.25,7);
    \node [font=\small] at (4.5,7) {2A};
    \node [font=\small] at (5.25,6.25) {5$\Omega$};
    \node [font=\small] at (6.75,6.75) {+};
    \node [font=\small] at (6.75,5.75) {-};
    \node [font=\small] at (6.75,6.25) {15V};
    \draw (7.25,6.5) to[short] (7.25,6);
    \draw (4,6.25) to[short] (4.5,6.25);
    \node at (5.75,7.25) [circ] {};
    \node at (5.75,5.25) [circ] {};
    \draw [->, >=Stealth] (4.25,6.75) -- (4.25,7);
    \node [font=\small] at (5.75,4.75) {(b)};
    \draw [->, >=Stealth] (0.25,6) -- (0.25,5.5);
    \draw (8,7.25) to[short] (8,6.5);
    \draw (8,6) to[short] (8,5.25);
    \draw (8,7.25) to[short] (9.25,7.25);
    \draw (9.25,7.25) to[european resistor] (9.25,5.25);
    \draw (8,5.25) to[short] (10.25,5.25);
    \draw (9.25,7.25) to[short] (10.75,7.25);
    \draw (10.75,7.25) to[short] (10.75,5.25);
    \draw (10.25,5.25) to[short] (10.75,5.25);
    \draw  (10.75,6.25) circle (0.25cm);
    \draw  (8,6.25) circle (0.25cm);
    \draw [->, >=Stealth] (8,6.5) -- (8,7);
    \draw (7.75,6.25) to[short] (8.25,6.25);
    \node [font=\small] at (8.25,7) {2A};
    \node [font=\small] at (8.75,6.25) {5$\Omega$};
    \node [font=\small] at (10.5,6.75) {+};
    \node [font=\small] at (10.5,5.75) {-};
    \node [font=\small] at (10.25,6.25) {15V};
    \node [font=\small] at (9.25,4.75) {(c)};
    \node at (9.25,7.25) [circ] {};
    \node at (9.25,5.25) [circ] {};
    \end{circuitikz}
    }%
    \end{figure}

\noindent 解:
(a)$P_r=-10V\Rightarrow P_r=-20W\quad
P_vs=15\times2W=30W\quad
P_cs=-5\times2W=-10W$\\电阻吸收功率$20W$
电压源消耗功率$30W$
电流源吸收功率$10W$\\
(b)$P_vs=-1\times15W=-15W\quad
P_cs=2\times15W=30W\quad
P_r=-3\times15W=-45W$\\电阻吸收功率$45W$
电压源发出功率$15W$
电流源发出功率$30W$\\
(c)$P_vs=-5\times15W=-75W\quad
P_cs=-30W\quad
P_r=45W$\\电阻吸收功率$45W$
电压源发出功率$75W$
电流源发出功率$30W$
\end{document}